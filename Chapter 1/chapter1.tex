\documentclass[spanish, a4paper, 12pt] {article}
\usepackage[spanish]{babel}
\usepackage[utf8]{inputenc}
\usepackage{amsmath}
\usepackage{amssymb}
\usepackage{amsfonts}
\usepackage{latexsym}
\usepackage{mathtools}
\usepackage{ebproof}
\usepackage{stmaryrd}
\usepackage[linguistics]{forest}
\usepackage{anysize}
%\marginsize{2cm}{2cm}{2cm}{3cm}
\newcommand\eqdef{\stackrel{\mathclap{\mbox{\tiny{def}}}}{=}}
\newcommand\eqac{\stackrel{\mathclap{\mbox{*}}}{=}}

\usepackage{graphicx}
\usepackage{hyperref}
\usepackage{float}
\usepackage{verbatim}
\usepackage[explicit]{titlesec}
\DeclareGraphicsExtensions{.pdf,.png,.jpg}

\titleformat{\subsection}{\normalfont\Large\bfseries}{}{0em}{Exercise \thesection.#1}

%\usepackage[a4paper,bindingoffset=0.2in,left=0.8in,right=0.8in,top=1.1in,bottom=1in,footskip=.25in]{geometry}

% MACROS
\newcommand{\aexp}[0]{\textbf{Aexp}}
\newcommand{\am}[0]{\textbf{AM}}
\newcommand{\bexp}[0]{\textbf{Bexp}}
\newcommand{\block}[0]{\textbf{Block}}
\newcommand{\code}[0]{\textbf{Code}}
\newcommand{\cont}[0]{\textbf{Cont}}
\newcommand{\decp}[0]{\textbf{Dec}{\scriptsize P}}
\newcommand{\decv}[0]{\textbf{Dec}{\scriptsize V}}
\newcommand{\Enve}[0]{\textbf{Env}{\scriptsize E}}
\newcommand{\Envp}[0]{\textbf{Env}{\scriptsize P}}
\newcommand{\Envv}[0]{\textbf{Env}{\scriptsize V}}
\newcommand{\enve}[0]{env_{\mbox{\scriptsize\textit{E}}}}
\newcommand{\envp}[0]{env_{\mbox{\scriptsize\textit{P}}}}
\newcommand{\envv}[0]{env_{\mbox{\scriptsize\textit{V}}}}
\newcommand{\ff}[0]{\textbf{ff}}
\newcommand{\FF}[0]{\textbf{FF}}
\newcommand{\loc}[0]{\textbf{Loc}}
\newcommand{\n}[0]{\textbf{N}}
\newcommand{\num}[0]{\textbf{Num}}
\newcommand{\p}[0]{\textbf{P}}
\newcommand{\pname}[0]{\textbf{Pname}}
\newcommand{\proc}[0]{\textbf{Proc}}
\newcommand{\pstate}[0]{\textbf{PState}}
\newcommand{\stack}[0]{\textbf{Stack}}
\newcommand{\state}[0]{\textbf{State}}
\newcommand{\stm}[0]{\textbf{Stm}}
\newcommand{\store}[0]{\textbf{Store}}
\newcommand{\T}[0]{\textbf{T}}
\newcommand{\TT}[0]{\textbf{TT}}
\newcommand{\Tt}[0]{\textbf{tt}}
\newcommand{\var}[0]{\textbf{Var}}
\newcommand{\while}[0]{\textbf{While}}
\newcommand{\z}[0]{\textbf{Z}}
\newcommand{\A}[0]{\mathcal{A}}
\newcommand{\B}[0]{\mathcal{B}}
\newcommand{\C}[0]{\mathcal{C}}
\newcommand{\D}[0]{\mathcal{D}}
\newcommand{\F}[0]{\mathcal{F}}
\newcommand{\M}[0]{\mathcal{M}}
\newcommand{\N}[0]{\mathcal{N}}
\renewcommand{\O}[0]{\mathcal{O}}
\renewcommand{\P}[0]{\mathcal{P}}
\renewcommand{\S}[0]{\mathcal{S}}
\newcommand{\Tcal}[0]{\mathcal{T}}
\newcommand{\Z}[0]{\mathcal{Z}}
\newcommand{\sam}[0]{\S_{\mbox{\scriptsize am}}}
\newcommand{\scs}[0]{\S_{\mbox{\scriptsize cs}}}
\newcommand{\sds}[0]{\S_{\mbox{\scriptsize ds}}}
\newcommand{\sns}[0]{\S_{\mbox{\scriptsize ns}}}
\newcommand{\ssos}[0]{\S_{\mbox{\scriptsize sos}}}
\newcommand{\stns}[0]{\rightarrow}
\newcommand{\staexp}[0]{\rightarrow_{Aexp}}
\newcommand{\stbexp}[0]{\rightarrow_{Bexp}}
\newcommand{\stsos}[0]{\Rightarrow}
\newcommand{\stam}[0]{\rhd}
\newcommand{\st}[2]{\left\langle#1, #2\right\rangle}
\newcommand{\ste}[3]{\left\langle#1, #2, #3\right\rangle}
\newcommand{\ass}[2]{\left[ #1\mapsto#2 \right]}
\newcommand{\eval}[1]{\left\llbracket #1 \right\rrbracket}
% While syntax
\newcommand{\While}[0]{\textbf{While}}
\renewcommand{\while}[2]{\mbox{\textbf{while} } #1 \mbox{ \textbf{do} } #2}
\newcommand{\If}[3]{\mbox{\textbf{if} } #1 \mbox{ \textbf{then} } #2 \mbox{ \textbf{else} } #3}
\newcommand{\Skip}[0]{\mbox{\textbf{skip}}}
\renewcommand{\=}[0]{:=}
\renewcommand{\;}[0]{\mbox{; }}
\newcommand{\eq}[0]{=}

\setlength{\parindent}{0cm}

\begin{document}
\section{Introduction}
\subsection{1}
The following is a statement in \While:
$$y \= 1\; \while{\neg(x=1)}{(y \= y*x \; x \= x-1)}$$
It computes the factorial of the initial value bound to x (provided that it is positive), and the result will be the final value of y. Draw a graphical representation of the abstract syntax tree.\\
\begin{forest}
[$S$
    [$S$
        [$y$]
        [$\=$]
        [$a$ [$1$]]
    ]
    [;]
    [$S$
        [\textbf{while}]
        [$b$
            [$\neg$]
            [$b$
                [$a$ [$x$]]
                [$\eq$]
                [$a$ [$1$]]
            ]
        ]
        [\textbf{do}]
        [$S$
            [$S$
                [$y$]
                [$\=$]
                [$a$
                    [$a$ [$y$]]
                    [$*$]
                    [$a$ [$x$]]
                ]
            ]
            [;]
            [$S$
                [$x$]
                [$\=$]
                [$a$
                    [$a$ [$x$]]
                    [$-$]
                    [$a$ [$1$]]
                ]
            ]
        ]
    ]
]
\end{forest}

\subsection{2}
Assume that the initial value of the variable $x$ is $n$ and that the initial value of $y$ is $m$. Write a statement in $\While$ that assigns $z$ the value of $n$ to the power of $m$.\\

Give a linear as well as a graphical representation of the abstract syntax.\\
$$z \= 1\; \while{y > 0}{(z \= z * x\; y \= y - 1)}$$
\begin{forest}
[$S$
    [$S$
        [$z$]
        [$\=$]
        [$a$ [$1$]]
    ]
    [;]
    [$S$
        [\textbf{while}]
        [$b$
            [$\neg$]
            [$b$
                [$a$ [$y$]]
                [$>$]
                [$a$ [$0$]]
            ]
        ]
        [\textbf{do}]
        [$S$
            [$S$
                [$z$]
                [$\=$]
                [$a$
                    [$a$ [$z$]]
                    [$*$]
                    [$a$ [$x$]]
                ]
            ]
            [;]
            [$S$
                [$y$]
                [$\=$]
                [$a$
                    [$a$ [$y$]]
                    [$-$]
                    [$a$ [$1$]]
                ]
            ]
        ]
    ]
]
\end{forest}
\subsection{4}
Suppose that the grammar for n had been
$$n\space::= 0 \,|\, 1 \,|\, 0 \, n \,|\, 1 \, n$$
Can you define $\N$ correctly in this case?\\

Of course, we can. We define $\N$ as follows:
\begin{align*}
    \N\eval{0} &= 0\\
    \N\eval{1} &= 1\\
    \N\eval{0\,n} &= \N\eval{n}\\
    \N\eval{1\,n} &= 2^{length(n)} + \N\eval{n}\\
\end{align*}
where $length(n)$ is the length of the numeral $n$ in digits.
\subsection{8}
Prove that the equations of Table 1.1 define a total function $\A$ in $\aexp \rightarrow (\state \rightarrow \Z$): First argue that it is sufficient to prove that for each $a \in \aexp$ and each $s \in \state$ there is exactly one value $v \in \Z$ such that $\A\eval{a}s = v$. Next use structural induction on the arithmetic expressions to prove that this is indeed the case.\\

A total function is a function where each element in the domain is mapped to an element in the codomain. In this case, we need to prove that for each arithmetic expression we have a function that maps a state into a numeral, in other word, if we have $a \in \aexp$ and $s \in \state$ we need to prove that there is only one value that is the result of the evaluation of $a$ in $s$. We name this value $v$.\\

In order to prove this result, we can use structural induction on the shape of the arithmetic expression. First, we check if the property holds for the basis elements:
\begin{itemize}
    \item If $a := n$ where $n \in \num$, we have $\A\eval{a}s = \N\eval{n}$ which depends only on $n$ and the function $\N$ which is a total function, so, there is no problem.
    \item If $a := x$ where $x \in \var$, we have $\A\eval{a}s = s\,x$. $s$ is itself a total function that provides only one result for each variable, so, again, we have only one value.
\end{itemize}
Now, we try to prove it for the composite elements:
\begin{itemize}
    \item If $a := a_1 + a_2$ where $a_1, a_2 \in \aexp$, we have $\A\eval{a}s = \A\eval{a_1}s + \A\eval{a_2}s$ and, assuming that the property holds for $a1$ and $a2$, because they are the immediate constituents of $a$, and knowing that the addition is a total function itself, the property holds in this case.
    \item The proof for $a := a_1 * a_2$ and $a := a_1 - a_2$ is analogous.
\end{itemize}
\subsection{9}
Assume that $s\,x = 3$, and determine $\B\eval{\neg(x = 1)}s$.\\

Ok, we have the following situation:
\begin{itemize}
    \item $\B\eval{(x = 1)}s = \ff$ because, using the third rule, we have that $\A\eval{x} = 1$ and $\A\eval{3} = 3$ and they are not equal.
    \item $\B\eval{\neg(x = 1)}s$ depends on the evaluation of $\B\eval{(x = 1)}s$ which is $\ff$, so, using the fifth rule, we obtain $\B\eval{\neg(x = 1)}s = \Tt$.
\end{itemize}
\subsection{10}
Prove that Table 1.2 defines a total function $\B$ in $\bexp \rightarrow (\state \rightarrow \T)$.\\

We must follow the same steps that we used in exercise 8. For each $b \in \bexp$ and $s \in \state$ we should obtain only one result when we evaluate $b$ in $s$. First, we check the property for the basis elements and then for the composite elements:
\begin{itemize}
    \item If $b := true$, we have $\B\eval{b}s = tt$. Obviously, we obtain only one result. The same holds when $b := false$.
    \item If $b := (a_1 = a_2)$ where $a_1, a_2 \in \aexp$, we must check whether or not $\A\eval{a1}s$ is equal to $\A\eval{a2}s$. Assuming that we have already prove that $\A$ is a total function, we have that the evaluation of $b$ in $s$ has only one result depending on s. We can use the same argument with $\le$.
    \item If $b = \neg b_1$ where $b_1 \in \bexp$, $\B\eval{b_1}s$ could be $\Tt$ or $\ff$, but only one of them, because we suppose that the property holds for $b_1$. In this case, using the fifth rule, we have that $\B\eval{b}s$ maps to only one value (the opposite of $\B\eval{b_1}s$).
    \item If $b = b_1 \wedge b_2$ where $b_1, b_2 \in \bexp$, we have $\B\eval{b_1}s = v_1$ and $\B\eval{b_1}s = v_2$ where $v_1$ and $v_2$ belong to $\T$ and are the only values returned by the evaluation of $b_1$ and $b_2$ in $s$. $\B\eval{b}s = \Tt$, if $v_1$ and $v_2$ are both $\Tt$, and $\B\eval{b}s = \ff$ otherwise. In both cases, there is only one possible value.
\end{itemize}
\end{document}
